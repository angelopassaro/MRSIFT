\documentclass[12pt,a4paper,oneside]{report}
\usepackage{main}
%Begin Title
\title{
  \textbf{SIFT Object Detection in Ambiente Distribuito con Apache Hadoop}\\
  {\large Università degli Studi di Salerno}\\
  {\large Corso di Sistemi Operativi Avanzati}\\
  {\large Anno Accademico 2017-2018}\\
  \vspace{0.5cm}
  {\includegraphics[width=5cm, height=5cm]{logo_standard.png}}
}
\author{Diego Avella, Angelo Passaro, Antonio Addeo.\\{\small \textbf{Supervisore:} Prof.Giuseppe Cattaneo}}
\date{\today}
\begin{document}
  \maketitle
  %Begin Abstract
  \begin{abstract}
    Ci troviamo in una situazione storica, per quanto riguarda il mondo dell'informatica, dove enormi quantità di dati (nell'ordine dei terabyte o addirittura dei petabyte e in un  futuro non molto lontano di zettabyte) vengono processati. Questi dati devono subire analisi e trasformazioni in modo da poter essere utilizzati in settori che non riguardano solamente il mondo informatico: ambienti scientifici come l'ingegneria, la biologia, e la medicina sono i maggiori consumatori di queste informazioni e hanno bisogno di risolvere i loro "problemi" in modo semplice e il più velocemente possibile. Proprio per questo il calcolo parallelo in ambiente con memoria condivisa (shared memory) non è più grado di poter gestire e sostenere queste enormi moli di dati senza sorpassare con facilità i limiti di tempo (CPU) e di spazio (Ram e Disco) per poter rispettivamente calcolare i risultati e memorizzarli. La nascita del paradigma Map-Reduce ideato da Google e del suo file system (GFS), reso famoso dalla Apache Foundation grazie rispettivamente alle implementazioni di Hadoop e HDFS, ha risolto una grandissima branca di problemi disparati senza dover ricorrere a soluzioni artificiose che tentano di ottimizzare a livelli estremi perdendo di portabilità ma che inoltre necessitano di dover essere continuamente aggiornate per poter andare incontro ai limiti fisici di cui si è parlato prima. L'obiettivo di questa tesi è quindi quello di dimostrare che un approccio distribuito di calcolo, attraverso il modello basato su Grid Computing, permette di "superare" questi limiti in maniera semplice e con risultati accettabili se non migliori che potenzialmente possono scalare con facilità. La tesina si propone di implementare un algoritmo parallelo di Object Detection tramite estrazione di caratteristiche utilizzando l'algoritmo SIFT di Lowe. Il primo capitolo tratterà delle basi su cui si fonda il calcolo distribuito e il Grid Computing, il secondo parlerà delle componenti fondamentali che hanno permesso la nascita dei moderni centri di calcolo sia dal punto di vista hardware che software. Il terzo capitolo introduce il calcolo distribuito tramite Apache Hadoop e il modello Map-Reduce spiegandone il funzionamento interno, il quarto applica le conoscenze acquisite per implementare l'applicativo parallelo e calcolarne le prestazioni e infine ci saranno delle conclusioni finali.
  \end{abstract}
  \tableofcontents
  \listoffigures
  %Start Thesis
  \chapter{Fondamenti di Calcolo Distribuito}
  In questo capitolo verranno introdotte le definizioni e i principi generali su cui si fonda il calcolo parallelo moderno. Verranno introdotti concetti fondamentali come granularità e parallelismo, si analizzeranno le infrastrutture di calcolo parallelo in ambiente distribuito più utilizzate ovvero Cloud Computing e Grid Computing e si presenteranno le limitazioni che il modello shared memory possiede rispetto al modello distribuito ripercorrendo la storia dell'informatica focalizzandoci sul destino che hanno subito le macchine parallele.
\section{Definizione di Calcolo Distribuito}
Il \textbf{Calcolo Distribuito} rappresenta un ramo dell'informatica che si occupa di studiare i \textbf{Sistemi Distribuiti}. Con quest'ultima definizione, facciamo riferimento ad un insieme di calcolatori che interagiscono tra loro (di solito tramite l'ausilio della rete) al fine di effettuare dei calcoli per concorrere ad un obiettivo comune che può essere un'erogazione di un servizio, il risultato di un algoritmo molto complesso etc. Un programma scritto per un ambiente distribuito è detto \textbf{software distribuito} e la programmazione distribuita è il processo di scrittura di questi software. Il protocollo con cui i calcolatori comunicano tra loro può essere di vario tipo come ad esempio HTTP, RPC, o basato su scambio di messaggi. Il calcolo distribuito è nato sia grazie alla nascita di Internet, che ha permesso a macchine remote di comunicare tra loro, sia dovuto ad altre cause:
\begin{enumerate}
  \item \textbf{Il limite di potenza di calcolo da parte dei processori:} Se fino alla fine degli anni 90 l'incremento della frequenza di clock di un processore aumentava in maniera costante, permettendo così agli algoritmi sequenziali di migliorare le loro prestazioni ogniqualvolta usciva hardware migliore, con l'avvento delle architetture multi core si è raggiunto un limite difficilmente superabile.
  \item \textbf{L'enorme mole di dati da dover processare:} Da quando si è integrato l'ausilio dei computer anche in settori estranei all'informatica, gli addetti al mestiere si sono trovati a dover risolvere problemi che avevano come input enormi quantità di dati che non potevano nemmeno lontanamente entrare nella memoria di un unico computer oltre a non avere la potenza computazionale per poter restituire risultati in tempi umanamente accettabili. Se a questo aggiungiamo l'avvento dei Big Data e di tutte quelle branche di analisi dei dati che si sono diffuse a causa di quest'ultimo, possiamo tranquillamente affermare che risulta impossibile per un singolo computer poter processare il tutto. 
\end{enumerate}
\section{Granularità}
%TODO trattare meglio la granularità
La granularità rappresenta "l'intervallo che occorre a due processi per sincronizzarsi misurato in numero di istruzioni macchina"\cite{cattaneo88}. Da questa definizione possiamo sicuramente dedurre che la granularità è inversamente proporzionale ai dati su cui vogliamo operare: più il partizionamento dei dati è piccolo, più istruzioni saranno necessarie per sincronizzarsi e di conseguenza la granularità dei programmi sarà finissima, viceversa se i dati su cui operare è grande sono necessarie meno sincronizzazioni e quindi la granularità è più grossa (coarsed). Quello che offre, ad esempio, un'architettura basata su Grid-Computing è quella di operare su una granularità più grossa al fine di poter usufruire al meglio dell'hardware di ogni singolo nodo sfruttando la \textbf{data-locality} e sincronizzarsi quando è solo strettamente necessario.
\section{Parallelismo}
Il \textbf{parallelismo} rappresenta la capacità con cui è possibile eseguire del calcolo in parallelo. È una caratteristica che dipende strettamente dalla granularità ed entrambe permettono di classificare le architetture di calcolo tramite la tassonomia di Flynn:
\begin{description}
  \item[SISD:] La sigla sta per \textit{Single Instruction Single Data} e non è presente nessuna forma di parallelismo in quanto le operazioni sono svolte in maniera sequenziale basandosi sulla classica architettura di \textbf{Von Neumann}. Un limite di questa architettura è sicuramente quello della singola connessione tra processore e memoria che rappresenta un collo di bottiglia per le prestazioni.
  \item[SIMD:] La sigla sta per \textit{Single Instruction Multiple Data} e il parallelismo è ottenuto dal fatto che ci sono più CPU che processano più flussi di dati in parallelo;è molto utilizzata per il calcolo scientifico ed è stata l'architettura che ha caratterizzato le grandi macchine parallele come \textit{Cry} e \textit{Connection Machine}.
  \item[MISD:] La sigla sta per \textit{Multiple Instruction Single Data} e il parallelismo è dovuto al fatto che più processi o task lavorano su un unico flusso di dati.
  \item[MIMD:] La sigla sta per \textit{Multiple Instruction Multiple Data} e il parallelismo è dovuto al fatto che più processori lavorano su più flussi di dati ed è l'architettura su cui si basano i cluster moderni.
\end{description}
Se invece vediamo il parallelismo dal punto di vista del programmatore è possibile distinguere due paradigmi principali:
\begin{description}
  \item[Parallelismo Esplicito:]con questo tipo di approccio il programmatore deve esplicitamente tenere conto di come suddividere i task e decidere come e quando sincronizzare questi ultimi. I vantaggi che ne derivano sono sicuramente un maggior controllo da parte del programmatore di quale parti del codice sia necessario parallelizzare oltre ad essere più efficiente ma questo comporta come svantaggio un maggiore sforzo da parte sua nel progettare la soluzione al problema. Un esempio di parallelismo esplicito è quando si usano librerie come MPI.
  \item[Parallelismo Implicito:] con questo tipo di approccio il programmatore scrive implicitamente del codice parallelo senza che lui ne sia effettivamente consapevole in quanto c'è chi si occupa di farlo per lui (un compilatore, un framework, un middleware). Gli svantaggi sono una perdita di controllo da parte del programmatore sul proprio codice ma col vantaggio che può focalizzarsi sulla sua risoluzione aumentando di molto la produttività. Un esempio di approccio implicito è quando si usano framework come Hadoop o Spark.
\end{description}
\section{Modelli di Calcolo Distribuito}
Per parlare di Calcolo distribuito e di Sistemi distribuiti dobbiamo sicuramente tenere conto dell'infrastruttura sottostante che sorregge l'intero sistema affinché possa funzionare al meglio. Ad oggi sono in uso tantissimo due tipologie di infrastrutture: \textbf{Cloud-Computing} e \textbf{Grid-Computing}.
\subsection{Cloud-Computing}
In un'architettura basata sul Cloud-Computing, le risorse informatiche, di qualunque tipo e non solo computazionali, vengono fornite all'utente quando richiesto. In un'infrastruttura come il Cloud i computer che erogano i servizi non è detto che siano situati sulla medesima rete locale (anzi è un caso raro) e questo comporta comunque un sovraccarico per quanto riguarda la comunicazione remota tra i vari nodi. Esistono diversi tipi di servizi che il Cloud Computing mette a disposizione:
\begin{description}
     \item[SaaS:] (Software as a Service) l'infrastruttura cloud mette a disposizione tutto il necessario per accedere ed usufruire di un software installato su di una macchina remota.
     \item[Paas:] (Platform as a Service)l'infrastruttura cloud mette a disposizione una piattaforma su cui è possibile installare eseguibili, librerie, programmi ed è a carico dell'utente gestire tutto il sistema. Questa tipologia di servizio è molto famosa e messa a disposizione da grandi big come Amazon (AWS) e Microsoft (Azure).
     \item[Iaas:] (Infrastructure as a Service)l'infrastruttura cloud mette a disposizione dell'utente oltre alle risorse virtuali anche quelle hardware che si dovrà gestire. 
   \end{description}   
\begin{figure}[!ht]
  \begin{center}
    \includegraphics[width=\linewidth]{cloud.jpg}
    \caption{I Servizi Cloud}
    \label{fig:cloud}
  \end{center}
\end{figure}
\subsection{Grid-Computing}
In un'architettura basata sul Grid-Computing, le risorse informatiche(anche qui di qualunque tipo ma sopratutto di calcolo e memorizzazione) possono essere messe a disposizione dell'utente per risolvere determinati problemi (di solito di natura molto complessa dal punto di vista del tempo e dello spazio). La maggiore differenza con il Cloud la si evince su come è costruita l'infrastruttura dei nodi. I nodi come dice stesso la parola formano una specie di "griglia" ma in quest'ultima la rete che collega i vari nodi è formata da una rete locale e questo comporta un migliore aumento delle prestazioni a causa del basso consumo a livello comunicativo che i nodi devono effettuare per spedire i dati attraverso la rete. Questa infrastruttura è quella che viene comunemente chiamata \textbf{data center} e vari sono stati negli anni le possibili soluzioni architetturali che potessero permettere di aggiungere nuovo hardware con facilità (quindi facilitare la scalabilità delle risorse), facili da gestire in fase di manutenzione e installazione del sistema operativo il tutto senza compromettere le prestazioni. 
\section{Modello Distribuito e Modello Shared Memory}
La domanda che ci si pone immediatamente è la seguente: Perché eseguire calcoli in ambiente distribuito piuttosto che in shared memory? La risposta la si ottiene se si guarda indietro alla storia dell'informatica e del calcolo parallelo. I primi modelli di macchine parallele sono state ideate tra gli anni cinquanta e sessanta per poi raggiungere il loro periodo d'oro tra gli anni settanta e ottanta con l'avvento d macchine come Cry-1, Cry-2, Encore che avevano un enorme potenza oltre all'enorme quantità di denaro necessario per comprarle e manutenerle. Queste macchine ormai sono un ricordo lontano e la causa del loro fallimento è riconducibile a vari fattori:
\begin{description}
    \item [Fisico:] Come ci dice la "Legge" di Moore il numero di transistor all'interno di un processore raddoppia all'incirca ogni 18 mesi. Sappiamo però che questo incremento non sarà, se non è già, più possibile perché non si può ridurre la grandezza dei transistor oltre una certa soglia (14nm anche se sappiamo che Intel è riuscita ad arrivare fino ad 11nm) per non provocare problemi legati al \textbf{\textit{thermal noise}}. 
    \item [Sviluppo:] Aumentare il grado di multiprogrammazione tramite l'ausilio di thread ha sicuramente arginato in parte il problema ma non è sicuramente una soluzione in quanto i limiti imposti dalle architetture shared memory rappresentano comunque un collo di bottiglia che inficia sulle prestazioni. Per tentare di risolvere questo problema allora si è cercato di sfruttare modelli di calcolo parallelo tramite l'ausilio di standard di comunicazione (come ad esempio MPI) che sono una possibile soluzione ma rappresentano uno sforzo non da poco da parte del programmatore che deve esplicitamente decidere quali parti del proprio programma parallelizzare o meno.
    \item[Portabilità:] Macchine del genere avevano sistemi operativi dedicati ed era impossibile effettuare porting dei programmi paralleli su altri architetture dovuto anche alla grande quantità di famiglie di processori che, prima che Intel ed AMD conquistassero il mercato, esistevano e il passaggio ad architetture multi core peggiorò la situazione portando all'estinzione non soltanto queste macchine ma anche un intera famiglia di sistemi operativi Unix-like
  \end{description}

  \chapter{Architettura di Calcolo Distribuito}
  \section{Datacenter}
Il \textbf{Datacenter} è una struttura composta da calcolatori che comunicano in rete, che aziende ed altre organizzazioni usano per organizzare, processare, memorizzare e disseminare enormi ammontare di dati. Essi rappresentano il cuore del business di un'organizzazione ed è necessario che soddisfino caratteristiche come scalabilità, sicurezza ed affidabilità oltre a possedere tutta l'attrezzatura necessaria come ad esempio la ventilazione, gruppi di continuità, sistemi di raffreddamento e naturalmente lo spazio necessario per installare il tutto. L'infrastruttura di un datacenter è cambiata molto durante gli anni e possiamo classificarla in tre categorie: 
\begin{description}
  \item[Infrastruttura Legacy:] In questa disposizione l'hardware è formato da tante commodity machines ma con l'obbligo di dover gestire ogni macchina separatamente ovvero si è costretti a dover installare su ogni macchina il sistema operativo, i programmi e le librerie necessarie. Da un primo impatto si può facilmente capire che con un numero molto grande di macchine la cosa risulta praticamente ingestibile e infatti questo modello è stato subito sostituito a favore delle architetture convergenti.
  \item[Infrastruttura Convergente:] In questa disposizione l'hardware è formato da enormi centri di calcolo che a differenza della struttura legacy sono gestiti tramite l'ausilio di un hypervisor quindi favorendo la gestione delle macchine ma con dei limiti dal punto di vista delle prestazioni in quanto la comunicazione con i dispositivi di memorizzazione avviene tramite una SAN\footnote{Storage Area Network} che rappresenta un collo di bottiglia nell'accedere e scrivere ai dati su disco.
  \item[Infrastruttura Iperconvergente:] Rappresenta la soluzione moderna per offrire un centro di calcolo performante. Questa architettura riprende il meglio delle due precedenti in quanto fa uso di commodity hardware, che costa poco, utilizza un hypervisor per gestire il datacenter ma la memorizzazione dei dati non avviene tramite una SAN. Quello che accade è che ogni rack possiede un insieme di piani che permette di aggiungere una motherboard formata da cpu, ram, e disco in modo che i dati delle singole motherboard possono accedere in maniera locale ai propri dischi aumentando le prestazioni e inoltre favorisce la scalabilità in quanto basta aggiungere al rack nuove motherboard quando è strettamente necessario.  
\end{description}
\begin{figure}[ht]
  \begin{center}
    \includegraphics[width=\linewidth]{hyperconvergence.jpg}
    \caption{Le tre architetture}
    \label{hyp}
  \end{center}
\end{figure}
\section{Hypervisor}
%TODO parlare in dettaglio dell vmm aggiungere citazione DISCO(seminario antonio Tino)
Ogni commodity hardware che forma il moderno datacenter non si limita all'esecuzione di un unico sistema operativo in quanto questo rappresenterebbe uno spreco di risorse di memoria e di CPU che possono essede dedicate per eseguire altro calcolo. La virtualizzazione è stata la tecnologia fondamentale che ha permesso alle architetture dei datacenter di evolversi da legacy a convergenti. L'hypervisor rappresenta il componente software o hardware necessario per ottenere virtualizzazione e quando Bugnon nel 97 presentò Disco\cite{bugnon97} e il VMM (sta per virtual machine monitor ed è un sinonimo di hypervisor) nessuno avrebbe mai pensato che la sua intuizione lo avrebbe portato al successo fondando VMware.
\section{Gestione dello storage}
Essendo che un datacenter deve processare e memorizzare ordini di petabyte, quindi che non entrano in un disco commerciale, è stato necessario trovare delle soluzioni adeguate. La storia ci insegna che inizialmente i dati venivano memorizzati in enormi dischi rigidi (ricordiamo ad esempio SLED di IBM) che non solo costavano tantissimo ma occupavano tanto spazio con altezze che arrivavano addirittura a 14 pollici. La soluzione è arrivata nel 1988 grazie a Patterson e alla sua architettura \textbf{RAID} che ha evidenziato con i suoi esperimenti\cite{patterson88} che è possibile ottenere, mettendo insieme un array di dischi economici con appositi controller, più memoria, migliori prestazioni e tolleranza ai guasti.
\subsection{RAID}
L'acronimo sta per \textit{Redundant Array of Inexpensive Disks} e rappresenta l'innovazione che ha permesso di raggruppare diversi dischi dando la sensazione al sistema che possano essere utilizzati come se fossero un unico volume. L'implementazione del RAID può essere effettuata o hardware o software dove nel primo caso si usano controller hardware fatti a hoc molto costosi (più di tutti i dischi messi assieme), nel secondo caso il tutto viene gestito dal sistema operativo con un normale controller (che può essere SATA, ATI, SCSI, in fibra). Il tipo di array viene identificato dal livello RAID che determina il numero di dischi minimo necessario per poter essere configurato. La caratteristica fondamentale di questa tecnologia è quella della ridondanza che permette di individuare e correggere errori ed è ottenuta con differenti tecniche che variano a seconda del livello di RAID utilizzato:
\begin{description} 
  \item[RAID 0:]Questo livello non possiede ridondanza e utilizza lo \textbf{striping} (unità minima in cui viene diviso ogni file per la scrittura) per distribuire i file nei dischi facendo si che le letture e scritture avvengano in parallelo. Ha come svantaggio naturalmente la perdita totale dei dati in caso di rottura del disco.
  \item[RAID 1:]Questo livello dedica fa sì che alcuni dischi vengano usati come copia per i dati in modo da intervenire in caso di guasto. Dal punto di vista delle prestazioni siamo pari a quelle di un singolo disco ma è presente la fault tolerance.
  \item[RAID 2:] Questo livello presenta delle caratteristiche simile al livello 1 ma con l'aggiunta di codici di correzione ECC dei dati. Questa configurazione è caduta in disuso a causa del fatto che ora i dischi attuali implementano di suo questa tipologia di correzione.
  \item[RAID 3:] Questo livello utilizza sia lo striping che il controllo della parità. Lo striping viene applicato a livello di segmenti e la parità mantiene le informazioni necessarie per poter recuperare i dati persi. In questo livello le scritture peggiorano poichè ad ogni scrittura si affianca il calcolo della parità e inolre la scrittura di essa avviene in un unico disco causando un collo di bottiglia sulle prestazioni totali.
  \item[RAID 4:] Questo livello utilizza le stesse funzionalità del livello 3 con la differenza che lo striping non viene effettuato con i segmenti ma a livello di blocchi.
  \item[RAID 5:] Questo livello utilizza le stesse funzionalità del livello 4 ma con la differenza che in questa configrazione non esiste un unico disco per la scrittura della parità ma su tutti vengono scritti dati o calcolo di parità (da notare che la parità non viene scritta sullo stesso disco dei dati). RAID 5 ha delle buone prestazioni che tendono a migliorare con l'aumento dei dischi installati.
  \item[RAID 6:] Questo livello ha le medesime caratteristiche del livello 5 con la differenza che effettua un doppio calcolo della parità (tramite codici di Solomon). Le prestazioni sono le medesime di RAID 5 con la presenza di una ridondanza aggiuntiva dei dati di controllo a causa della parità.
  \item[RAID Annidati:] Sono combinazioni di configurazioni di RAID permettendo così di accorpare le caratteristiche dei livelli. Le due combinazioni più diffuse sono la 01 e la 10. La prima prende due configurazioni RAID 0 e le combina in un RAID 1 e questo comporta che ogni gruppo di dischi conterrà la copia speculare dell'altro gruppo. Il secondo prende gruppi di dischi in RAID 1 e si combinano in RAID 0 permettendo così di vedere il tutto come se fosse un unico disco e inoltre permette la tolleranza del guasto di sue dischi.
\end{description}
\subsection{Software Defined Storage}
%TODO argomentare meglio di cosa si tratta e spiegare meglio Ceph
Ceph è una piattaforma di storage distribuito creata da Red Hat recentemente (2016) e rappresenta quella che potrebbe essere in futuro una valida alternativa a RAID per gestire la memorizzazione dei dati. Questa piattaforma fornisce intefacce di storage di diverse tipologie (a oggetti, a file) quindi permettendo una grande flessibilità ma soprattutto riesce a scalare nell'ordine degli exabyte. La replicazione dei dati avviene al livello software rendendo così Ceph una piattafroma indipendente dall'hardware. La ragione del suo successo risiede nel fatto che sia possibile accedere ai dati in maniera completamente trasparente e diversificata permettendo di adattarsi alle esigenze delle organizzazioni. Un'altra caratteristica a fvore è che un sistema Ceph può essere costruito con commodity hardware riducendo di molto i costi a discapito di un controller RAID hardware.
\section{Sistema Operativo}
%TODO Aggiungere dettagli. Storia dei sistemi operativi SPIN, TORNADO.
Qui si parla dell'importanza del sistema operativo in ambiente distribuito.
\subsection{Kernel}
%TODO aggiungere microkernel, monolitico ed exokernel
Qui si parla delle varie architetture kernel.
\section{File System} 
In un data-center è indispensabile avere un filesystem che sia trasparente e che nasconda all'utente che usufruisce del cluster l'effettiva locazione dei file sui dischi. L'informatica ha conosciuto due diverse filosofie di implementazioni di filesystem su reti: I primi tentativi concreti sono stati ottenuti con l'introduzione del protocollo \textbf{SMB} (Server Message Block) ed è ancora tuttora utilizzato da Windows e da Linux (Samba), successivamente negli verso la metà degli anni 80 alla Carnagie Mellow University venne prototipizzato \textbf{AFS} (Andrew File System) che nonostante le implementazioni artificiose che i ricercatori dovettero fare per sistemare alcuni problemi di scalabilità e gestione dei fault, fu la base per la nascita negli anni 90 di CODA. Una prospettiva completamente ortoginale ad AFS è stata NFS ideato da Sun che.
%% Aprire tutta una parte sui file system
\subsection{AFS}
\subsection{NFS}
\subsection{Coda}
\subsection{GFS}

  \chapter{Calcolo Distribuito con Hadoop e Map-Reduce}
  \section{Storia di Hadoop}
Il framework Hadoop fu creato da \textbf{Doug Cutting}, creatore di \textbf{Apache Lucene}, la libreria più utilizzata per quanto riguarda la ricerca di tipo testuale. Questo framework fonda le sue radici da \textbf{Apache Nutch}, una motore di ricerca per il web open source a sua volta parte integrante del progetto Lucene. Scrivere un intero motore di ricerca da zero era un obiettivo molto ambizioso. Il progetto Nutch partì nel 2002 ed ebbe un buon successo tuttavia i creatori si resero immediatamente conto che la loro architettura non sarebbe stata in grado di scalare a sufficienza a causa dell'alto numero di pagine web da indicizzare (già allora ce ne erano più di un miliardo). Nel 2003 Google pubblicava il paper in cui introduceva il GFS e Nutch intuì che questo filesystem avrebbe risolto i problemi di memorizzazione e amministrazione dei dati cheavevano e decisero di implementarne una versione open source che venne chiamata NDFS (Nutch Distributed File System). Un anno dopo smepre Google pubblicò il paper che introdusse il paradigma Map-Reduce e ancora una volta il creatore capì che questa idea lo avrebbe aiutato a risolvere i problemi di Apache Nutch e nel 2005 il suo team aveva già implementato una sua versione open source e nel giro di 6 mesi tutti gli algoritmi del motore di ricerca furono adattati per essere eseguiti su NDFS e usando Map-Reduce. Questa "accoppiata" ebbe così tanto successo che nel Febbraio 2006 Nutch decise di spostarlo in un progetto indipendente chiamato Hadoop e nello stesso periodo Doug Cutting si unì a Yahoo! che forni un team dedicato e risorse finanziarie che trasformare Hadoop in un sistema che potesse essere eseguito per scalare anche sul web e ci riuscirono nel 2008 quando l'azienda annunciò che il suo indice di ricerca era stato generato da un cluster Hadopo da 10000 core. Nel Gennaio 2008 Hadoop divenne il progetto di punta della fondazione Apache ed è tuttora utilizzato da grandi compagnie come Facebook, New York Times e finnziato dai big dell'informatica come IBM, Microsoft e dalla stessa Google.
\section{Architettura di Hadoop}
qui si parla in breve dell'architettura di hadoop
\subsection{HDFS}
HDFS è il filesystem distribuito che utilizza il framework Hadoop per processare i dati
qui si parla di yarn
\section{Paradigma Map-Reduce}
\subsection{Map}
Qui si parla del metodo map
\subsection{Reduce}
qui si parla del metodo reduce
\section{Come funziona un Job Map Reduce}

  \chapter{Utilizzo di Hadoop per Object Detection }
  \section{Descrizione del problema}
Viene analizzato il problema
\section{Versione Sequenziale}
Descrizione della risoluzione del problema in ambiente shared memory.
\section{Versione Distibuita con Hadoop}
Descrizione della versione distribuita con Hadoop
\section{Analisi delle prestazioni}
%TODO fare weak scaling e strong scaling
Vengono forniti i benchmark con rispettive spiegazioni

  \chapter{Conclusioni}
  La Computer Vision è un settore molto complesso: Questo primo prototipo di Object Detection parallelo mostra le sue lacune ma anche dei spunti interessanti. Ci rendiamo perfettamente conto che i risultati non sono dalla nostra parte ma possiamo parzialmente giustificarli a causa della nostra inesperienza in entrambi i settori: La mancanza di librerie ottimizzate per lavorare con le immagini in Hadoop, ci ha costretto a scrivere un piccolo strato di codice necessario per trattare le immagini più agevolmente ma purtroppo non ottimizzato a supportare decenti carichi di lavoro. Anche la size del dataset ha implicato molto sui risultati generando un sovraccarico di comunicazione non necessario ma che abbiamo tenuto opportuno lasciare così perché immagini con alta risoluzione richiedono enormi matrici di float non facilmente gestibili in memoria e che avrebbero rallentano di molto l'applicativo durante il passaggio dei dati tra i map e i reduce task. Un altro fattore che ha sicuramente inficiato è il continuo leggere dall'HDFS per iterare su tutte le immagini ed effettuare il calcolo dei match. Nonostante queste lacune riteniamo l'applicativo molto interessante in quanto un approccio di calcolo distribuito basato su Hadoop riesce comunque ad ammortizzare i tempi di esecuzione e siamo certi che uno studio più approfondito del framework e di Computer Vision permetta di migliorare sensibilmente l'algoritmo. Altro spunto di miglioramento potrebbe essere l'ausilio di un database distribuito per memorizzare le features e migliorare la probabilità di matching esatto ed avere un risultato più raffinato. Infine ci sentiamo di concludere che questa esperienza con Hadoop è stata molto illuminante ed esaltante: Mettere le mani sul cluster remoto, effettuare comissioning e decomissioning, valutare le prestazioni, adattare il framework alle nostre esigenze, Usare JNI per utilizzare OpenCV ci ha permesso di capire che questo settore richiede tantissime competenze ed una conoscenza trasversale del settore informatico. 
  {\RaggedRight
\begin{thebibliography}{9}
\addtolength{\leftmargin}{0.2in}
\setlength{\itemindent}{0.2in}
%Start Citations

  \bibitem{cattaneo88} G.Cattaneo, "The basic Hardware \& software to medium grain parallelism", 1988.
  
  \bibitem{bugnon97} E.Bugnion,S.Devine,K.Govil,M.Rosenblum, "Disco: Running Commodity Operating Systems on Scalable Multiprocessors", 1997.
  
  \bibitem{patterson88} D.Patterson,G.Gibson,K.H.Randy, "A Case for Redundant Arrays of Inexpensive Disks (RAID)", 1988.
  
  \bibitem{lowe04} David G. Lowe, "Distinctive Image Features from Scale-Invariant Keypoints", 2004.
  
  \bibitem{gamsa98} B.Gamsa,O.Krieger,J.Appavoo, M.Stumm, "Tornado: Maximizing Locality and Concurrency in a Shared Memory Multiprocessor Operating System", 1998.
  
  \bibitem{bershad95} B.Bershad,S.Savage,E.Siren,D.Becker,C.Chambers,S.Eggers, "Extensibility, Safety and Performance in the SPIN Operating System", 1995.
  
  \bibitem{hartig97}H.H\"{a}rtig,M.Hohmuth,J.Liedtke,S.Sch\"{o}nberg,J.Wolter, "The Performance of $\mu$-Kernel-Based Systems", 1997.
  
  \bibitem{rashid88} R.Rashid,A.Tevanian,M.Young,D.Golub,D.Black,J.Chew, "Machine Indipendent Virtual Memory Management for Paged Uniprocessor and Multiprocessor Architecture" 1988.
  
  \bibitem{appel91} A.Appel,K.Li, "Virtual Memory Primitives for User Programs", 1991.
  
  \bibitem{engler94} R.Engler,M.Kasshoek,J.O'Toole, "Exokernel:An Operating System Architecture for Application-Level Resource Management", 1994.
	\bibitem{kay89} J.Kay, P.Lauder, "A Fair Share Scheduler", 1989.
	\bibitem{waldspurger94} C.A.Waldspurger, W.E.Weihl, "Lottery Scheduling: Flexible Proportional-Share Resource Management", 1994.
	\bibitem{rosenblum92} M.Rosenblum, J.K.Osterhout, "The Design and Implementation of a Log-Structured File System", 1992.
	\bibitem{howard88} J.H.Howard, M.L.Kazar, S.G.Menees,D.A.Nichols,M.Satyanarayanan, R.N.Sidebotham, M.J.West, "Scale and Performance in a Distributed File System", 1988.
	\bibitem{kistler92} J.J.Kistler, M.Satyanarayanan, "Disconneted Operation in the Coda File System", 1992.
	\bibitem{anderson96} E.T.Anderson, M.D.Dahlin,J.M.Neefe,D.A.Patterson,D.S.Roselli,R.D.Wang, "Serverless Network File System", 1996.
	\bibitem{ghemawat03} S.Ghemawat,H.Gobioff,S.Leung, "The Google File System", 2003.
	\bibitem{thekkat94} A.Thekkat, M.Levy, D.Lazowska, "Separating Data And Control Transfer in Distributed Operating Systems", 1994.
	\bibitem{dahlin94} D.Dahlin, R.Wang, T.E.Anderson, D.A.Patterson, "Cooperative Caching: Using Remote Client Memory to Improve File System Performance", 1994.
	\bibitem{petersen97} K.Petersen,M.Spreitzer,D.B.Terry, Marvin.M.Theimer,A.J.Demers, "Flexible Update Propagation for Weakly Consistent Replication", 1997.
	\bibitem{gibson98} G.A.Gibson, D.F.Nagle, K.Amiri, J.Butler, F.W.Chang, H.Gobioff, C.Hardin, E.Riedel, D.Rochberg, J.Zelenka, "A Cost-Effective, High Bandwidth Storage Architecture", 1998.
\end{thebibliography}
}

\end{document}
