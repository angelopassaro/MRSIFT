\section{Data Center}
Il \textbf{Data Center} è una struttura composta da calcolatori che comunicano in rete, che aziende ed altre organizzazioni usano per organizzare, processare, memorizzare e disseminare enormi ammontare di dati. Essi rappresentano il cuore del business di un'organizzazione ed è necessario che soddisfino caratteristiche come scalabilità, sicurezza ed affidabilità oltre a possedere tutta l'attrezzatura necessaria come ad esempio la ventilazione, gruppi di continuità, sistemi di raffreddamento e naturalmente lo spazio necessario per installare il tutto. L'infrastruttura di un data center è cambiata molto durante gli anni e possiamo classificarla in tre categorie: 
\begin{description}
  \item[Infrastruttura Legacy:] In questa disposizione l'hardware è formato da tante commodity machines\footnote{Macchine a basso costo} ma con l'obbligo di dover gestire ogni macchina separatamente ovvero si è costretti a dover installare su ogni macchina il sistema operativo, i programmi e le librerie necessarie. Da un primo impatto si può facilmente capire che con un numero molto grande di macchine la cosa risulta praticamente ingestibile e infatti questo modello è stato subito sostituito a favore delle architetture convergenti.
  \item[Infrastruttura Convergente:] In questa disposizione l'hardware è formato da enormi centri di calcolo che a differenza della struttura legacy sono gestiti tramite l'ausilio di un hypervisor quindi favorendo la gestione delle macchine ma con dei limiti dal punto di vista delle prestazioni in quanto la comunicazione con i dispositivi di memorizzazione avviene tramite una SAN\footnote{Storage Area Network} che rappresenta un collo di bottiglia nell'accedere e scrivere ai dati su disco.
  \item[Infrastruttura Iper-convergente:] Rappresenta la soluzione moderna per offrire un centro di calcolo performante. Questa architettura riprende il meglio delle due precedenti in quanto fa uso di hardware a basso costo, utilizza un hypervisor per gestire il data center ma la virtualizzazione è effettuata su ogni componente del centro di calcolo (SAN, Cpu, Ram, Rete). Questo permette una migliore gestione delle risorse hardware ed una capacità di aumentarne le prestazioni con facilità: un data center iper-convergente, è formato da tanti armadi contenenti tantissime schede madri ognuna con slot per aggiungere risorse di computazione o memorizzazione. Quando si necessità di aggiungere nuove risorse basta semplicemente aggiungere i pezzi necessari senza il minimo sforzo. 
\end{description}
\begin{figure}[H]
  \begin{center}
    \includegraphics[scale=0.6]{hyperconvergence.jpg}
    \caption{Le tre architetture}
    \label{fg:hyperconvergence.jpg}
  \end{center}
\end{figure}
\section{Hypervisor}
Ogni commodity hardware che forma il moderno data center non si limita all'esecuzione di un unico sistema operativo in quanto questo rappresenterebbe uno spreco di risorse di memoria e di CPU che possono essere dedicate per eseguire altro calcolo. La virtualizzazione è stata la tecnologia fondamentale che ha permesso alle architetture dei data center di evolversi da legacy a convergenti. L'hypervisor rappresenta il componente software o hardware necessario per ottenere virtualizzazione e quando Bugnon nel 97 presentò Disco\cite{bugnon97} e il VMM (sta per virtual machine monitor ed è un sinonimo di hypervisor) nessuno avrebbe mai pensato che la sua intuizione lo avrebbe portato al successo fondando VMware. Gli hypervisor si possono dividere in tre grandi categorie:
\begin{itemize}
	\item \textbf{Bare-Metal:} Si installa direttamente su di una macchina senza l'ausilio di un sistema operativo ospitante in modo che possa accedere direttamente all'hardware sottostante. La ragione principale per installare questa tipologia è per eseguire sistemi operativi multipli, senza sovraccarico da parte del sistema operativo ospitante ed è molto utilizzato in ambito server. Tipologie note in questo settore sono \textit{VMWare ESxi}, \textit{Citrix XenServer}, \textit{Microsoft Hyper-V}.
	\item \textbf{Hosted:} Richiede che ci sia un sistema operativo ospitante, che tratta i sistemi ospiti come se fossero normali processi. Virtual Box, VMWare sono esempi di questo tipo di tipologia.
	\item \textbf{Paravirtualization:} Un approccio alternativo richiede modificare il sistema operativo ospite affinché esegua delle system call all'hypervisor invece di eseguire delle istruzioni macchina che poi l'hypervisor deve simulare ed è chiamato \textbf{paravirtualizzazione}. 
\end{itemize} 
\section{Gestione dello storage}
Essendo che un data center deve processare e memorizzare ordini di petabyte, quindi che non entrano in un disco commerciale, è stato necessario trovare delle soluzioni adeguate. La storia ci insegna che inizialmente i dati venivano memorizzati in enormi dischi rigidi (ricordiamo ad esempio SLED di IBM) che non solo costavano tantissimo ma occupavano tanto spazio con altezze che arrivavano addirittura a 14 pollici. La soluzione è arrivata nel 1988 grazie a Patterson e alla sua architettura \textbf{RAID} che ha evidenziato con i suoi studi \cite{patterson88} che è possibile ottenere, mettendo insieme un array di dischi economici con appositi controller, più memoria, migliori prestazioni e tolleranza ai guasti.
\subsection{RAID}
L'acronimo sta per \textit{Redundant Array of Inexpensive Disks} e rappresenta l'innovazione che ha permesso di raggruppare diversi dischi dando la sensazione all'utente che possano essere utilizzati come se fossero un unico volume. L'implementazione del RAID può essere effettuata sia hardware sia software dove nel primo caso si usano controller hardware fatti a hoc molto costosi (più di tutti i dischi messi assieme), nel secondo caso il tutto viene gestito dal sistema operativo con un normale controller (che può essere SATA, ATI, SCSI, in fibra). Il tipo di array viene identificato dal livello RAID che determina il numero di dischi minimo necessario per poter essere configurato. La caratteristica fondamentale di questa tecnologia è quella della ridondanza che permette di individuare e correggere errori ed è ottenuta con differenti tecniche che variano a seconda del livello di RAID utilizzato:
\begin{description} 
  \item[RAID 0:]Questo livello non possiede ridondanza e utilizza lo \textbf{striping} (unità minima in cui viene diviso ogni file per la scrittura) per distribuire i file nei dischi facendo si che le letture e scritture avvengano in parallelo. Ha come svantaggio naturalmente la perdita totale dei dati in caso di rottura del disco.
  \item[RAID 1:]Questo livello fa sì che alcuni dischi vengano usati come copia per i dati in modo da intervenire in caso di guasto. Dal punto di vista delle prestazioni siamo pari a quelle di un singolo disco ma è presente tolleranza ai guasti (fault tolerance).
  \item[RAID 2:] Questo livello presenta delle caratteristiche simile al livello 1 ma con l'aggiunta di codici di correzione ECC\footnote{Sta per Error Correcting Code} dei dati. Questa configurazione è caduta in disuso a causa del fatto che ora i dischi attuali integrano questa tipologia di correzione.
  \item[RAID 3:] Questo livello utilizza sia lo striping che il controllo della parità. Il primo viene applicato a livello di segmenti e il secondo mantiene le informazioni necessarie per poter recuperare i dati persi. In questo livello le scritture peggiorano poiché ad ogni scrittura si affianca il calcolo della parità e inoltre la scrittura di quest'ultima avviene in un unico disco causando un collo di bottiglia sulle prestazioni totali.
  \item[RAID 4:] Questo livello utilizza le stesse funzionalità del livello 3 con la differenza che lo striping non viene effettuato con i segmenti ma a livello di blocchi.
  \item[RAID 5:] Questo livello utilizza le stesse funzionalità del livello 4 ma con la differenza che in questa configurazione non esiste un unico disco per la scrittura della parità ma su tutti vengono scritti dati o calcolo di parità (da notare che la parità non viene scritta sullo stesso disco dei dati). RAID 5 ha delle buone prestazioni che tendono a migliorare con l'aumento dei dischi installati.
  \item[RAID 6:] Questo livello ha le medesime caratteristiche del livello 5 con la differenza che effettua un doppio calcolo della parità (tramite codici di Solomon). Le prestazioni sono le medesime di RAID 5 con la presenza di una ridondanza aggiuntiva dei dati di controllo a causa della parità.
  \item[RAID Annidati:] Sono combinazioni di configurazioni di RAID permettendo così di accorpare le caratteristiche dei livelli. Le due combinazioni più diffuse sono la 01 e la 10. La prima prende due configurazioni RAID 0 e le combina in un RAID 1 e questo comporta che ogni gruppo di dischi conterrà la copia speculare dell'altro gruppo. Il secondo prende gruppi di dischi in RAID 1 e si combinano in RAID 0 permettendo così di vedere il tutto come se fosse un unico disco e inoltre permette la tolleranza del guasto di due dischi.
\end{description}
\subsection{Software Defined Storage}
Ceph è una piattaforma di memorizzazione distribuita creata da Red Hat recentemente (2016) e rappresenta quella che potrebbe essere in futuro una valida alternativa a RAID per gestire il salvataggio dei dati. Questa piattaforma fornisce interfacce di memorizzazione di diverse tipologie (a oggetti, a file) quindi permettendo una grande flessibilità ma soprattutto riesce a scalare nell'ordine degli exabyte. La replicazione dei dati avviene al livello software rendendo così Ceph una piattaforma indipendente dall'hardware. La ragione del suo successo risiede nel fatto che sia possibile accedere ai dati in maniera completamente trasparente e diversificata permettendo di adattarsi alle esigenze delle organizzazioni. Un'altra caratteristica a favore è che un sistema Ceph può essere costruito con hardware di basso costo a discapito di un controller RAID hardware. Ceph come molte altre realtà, rappresenta quello che oggi viene comunemente definito come \textit{Software Defined Storage} e nasce per andare incontro alle esigenze delle architetture iper-convergenti utilizzando la virtualizzazione per separare tutte le funzioni di memorizzazione dall'hardware.
\section{Sistema Operativo}
L'introduzione di tecniche efficienti per aumentare le prestazione dei dischi, non sono sufficienti per avere un centro di calcolo efficiente. Anche il sistema operativo gioca la sua parte e deve essere implementato affinché usi al meglio l'hardware sottostante. Questo aspetto non è da sottovalutare perché il sistema operativo gestisce servizi cruciali come lo scheduling dei processi, gestione della memoria virtuale, mappatura dei blocchi liberi, etc... Durante la storia dell'informatica la sistemistica è stato un ambito di ricerca e sviluppo che ha sempre portato ventate di novità questo grazie anche ad un ambiente universitario sempre competitivo e alla ricerca di nuove soluzioni che spalleggiassero la propria visione. Questo ha portato alla nascita di tantissimi sistemi, ognuno con delle caratteristiche particolari e che hanno gettato le basi per quelli moderni. Tra i più significativi è necessario citare:
\begin{description}
  \item[Tornado:] Tornado è stato uno dei primi sistemi operativi che scommise sul passaggio alle architetture multi core dei processori e si focalizzò tantissimo nello sfruttare al meglio questi ultimi massimizzandone le prestazioni sfruttando principi di località e tecniche di tipo Object Oriented.\cite{gamsa98}
  \item[SPIN:] SPIN è stato uno dei sistemi che ha visto nella programmazione ad oggetti un nuovo approccio per progettare sistemi operativi. Scritto completamente in un linguaggio ad alto livello come MODULA-3 vantava modularità, estensibilità senza compromettere di molto le prestazioni.\cite{bershad95}
  \item[System V:] L'unico sistema della famiglia Unix continuato ad essere sviluppato presso i laboratori AT\&T ed usato tantissimo in ambito commerciale. Sono state rilasciate ben 5 versioni e dominò per lungo tempo il mercato dei processori x86. L'avvento di popolarità di Linux e BSD ha causato il suo decesso e infatti la versione 6 che doveva essere rilasciata nel 2004 fu poi cancellata e pian piano sparì dalla circolazione. 
  \item[BSD:] Ultimo vero superstite originario della famiglia Unix è stato uno dei primi sistemi ad includere il protocollo TCP/IP ed è famoso per la sua stabilità, affidabilità e rispetto per gli standard. In giro sono rimaste unicamente le sue controparti opensource ovvero FreeBSD e OpenBSD.
  \item[Windows NT:] Se MSDOS può non essere considerato un sistema operativo vero proprio, il suo successore Windows NT ha portato molti cambiamenti e Microsoft è stata una delle prime ad aggiungere il supporto ai processori Intel, in particolare il 386 e 486, intuendo da lì a poco che il mercato delle CPU si sarebbe pian piano ridotto ad un insieme ristretto di concorrenti. Questo unito alla sua semplicità ne hanno fatto il sistema operativo de facto utilizzato dall'utenza media.
  \item[Mac:] Anche Apple e Steve Jobs capirono presto che la nuova fetta di mercato su cui incentrare il nuovo business sarebbero stati i personal computer e c'era bisogno di un sistema operativo semplice ed estensibile. Codice Sorgente di BSD e il microkernel Mach messi assieme diedero vita a quello che oggi chiamiamo MacOS. Apple fu uno dei pochissimi leader del settore informatico che credette nello sviluppo di un microkernel ma questa scelta  produsse i suoi frutti solo dopo tanti anni quando le CPU passaro da istruzioni a 32 bit a 64.
  \item[Gnu/Linux:] Richard Stallman stanco della piega con cui il software proprietario stava prendendo piede decise di fondare il movimento Free-Software implementando insieme a decine di altri programmatori al mondo un sistema operativo che assomigliasse a Unix ma che fosse libero. Lo sforzo di tantissimi volontari diede luce al sistema GNU che non essendo ancora provvisto di un kernel (la loro scelta era di scrivere un microkernel ma la difficoltà nello sviluppo di quest'ultimo era molto alta) non gli permise di vedere la luce fino a quando un giovane studente finlandese di nome Linus Torvalds non creò come tesi per il suo dottorato di ricerca un kernel monolitico che chiamò Linux. Fortuna volle che l'integrazione delle due parti fu molto semplice e anni di contributi da parte di tantissimi sviluppatori lo hanno reso il sistema operativo che oggi conosciamo: stabile, affidabile, robusto e praticamente compatibile con ogni tipologia di hardware esistente. Queste sue caratteristiche lo rendono il signore incontrastato in ambiente server e su piattaforma cloud.
\end{description}
Lo sviluppo però non fu solo dal punto di vista tecnico: anche dal punto vista algoritmico la scoperta di nuove soluzioni portarono al miglioramento dei sistemi operativi. Possiamo ricordare gli algoritmi di scheduling che sono stati progettati quando computer e mainframe passarono ma monoutente a multiutente come ad esempio il \textbf{Fair-Scheduler} basato sul sistema a quotazioni\cite{kay89} e il \textbf{Lottery Scheduling} il cui funzionamento prevedeva un meccanismo simile al gioco della lotteria in cui l'utente col "biglietto vincente" poteva acquisire le risorse.\cite{waldspurger94}
\subsection{Kernel}
Il componente fondamentale su cui si appoggia un sistema operativo è sicuramente il kernel. Esso si occupa dell'intera gestione dell'hardware fornendo dei meccanismi di base su cui si basano i servizi soprastanti del sistema. Proprio il numero di servizi di base che deve offrire il kernel è causa di un intero movimento che ha dato origine a vari approcci di tipo architetturale con cui deve essere sviluppato un kernel:
\begin{description}
  \item[Kernel Monolitico:] In un kernel di tipo monolitico, tutti i servizi del sistema operativo condividono la medesima area di memoria (ovvero tutto è all'interno del kernel space) del kernel ed eseguiti insieme ad esso. Il kernel espone i propri servizi tramite delle system call e permette un accesso performante all'hardware sottostante. Gli svantaggi possono essere di vario tipo: basta un unico errore in un modulo kernel per mandare in crash l'intero sistema, se il kernel comincia ad aumentare di dimensioni diventa troppo grande e soprattutto ingestibile da manutenere e soprattutto non sono portabili e quindi un passaggio potenziale ad una nuova architettura comporta la riscrittura di tutte le componenti.
  \item[Microkernel:] Il microkernel rappresenta come idea l'opposto di un kernel monolitico. L'idea alla base è di mantenere il kernel più piccolo possibile definendo unicamente le funzionalità fondamentali all'interno del kernel space (scheduling, gestione blocchi) e implementare all'interno dello user space gli altri servizi facendo sì che questi ultimi  interagiscano con il kernel attraverso scambio di messaggi tramite RPC. I vantaggi sono un kernel più compatto e più facile da manutenere, estensibilità ma al prezzo di un calo di prestazioni. C'è anche da sottolineare comunque che questo calo di prestazioni non è molto cospicuo come infatti hanno dimostrato gli studi su L4\cite{hartig97} e sui meccanismi di gestione della memoria virtuale\cite{rashid88,appel91}.
  \item[Kernel Ibridi:] Sono un ibrido tra un kernel monolitico e un microkernel e racchiudendo al suo interno qualche funzionalità in più di un normale microkernel.
  \item[Exokernel:] Sono un estremizzazione delle architetture kernel in cui il numero di astrazioni sull'hardware è ridotto all'osso permettendo agli sviluppatori di prendere le decisioni più appropriate e il kernel garantisce unicamente la gestione e messa in sicurezza delle risorse. Le applicazioni richiedono specifiche risorse fisiche (blocchi del disco, indirizzi di memoria) mentre il kernel assicura solo che le risorse siano disponibili e che l'applicativo abbia il diritto ad accederci \cite{engler94}.
\end{description}
\section{File System} 
In un centro di calcolo è indispensabile avere un filesystem distribuito che sia trasparente e che nasconda all'utente che usufruisce del cluster l'effettiva locazione dei file sui dischi o di come siano memorizzati. L'informatica ha conosciuto diverse filosofie di implementazioni di filesystem su reti a partire dai primi anni 70 con l'invenzione del primo filesystem ma i primi tentativi concreti sono stati ottenuti con l'introduzione del protocollo \textbf{SMB} (Server Message Block) ed è ancora tuttora utilizzato da Windows e da Linux (Samba). Alla nascita dei filesystem distribuiti odierni c'è anche da considerare tutto un intera branca di studi che ne ha ottimizzato le caratteristiche, ad esempio migliori algoritmi per la gestione dei dati in cache\cite{dahlin94}, gestione della replicazione dei dati\cite{petersen97}, architetture avanzate per la gestione dello storage\cite{gibson98}, e tecniche per migliorare la separazione tra dati e meta dati \cite{thekkat94}.
\subsection{AFS}
L'Andrew File system nato negli anni 80 è stato il precursore dei moderni file system distribuiti. Sviluppato alla Carnagie Mellow, presentava un' architettura client server (Venus e Vice) e vennero sviluppati diversi protitipi. L'idea di base è che il client Venus richieda al server Vice solamente l'ubicazione del file quando è necessario aprire o chiudere quest'ultimo mentre per effettuare delle modifiche come ad esempio delle scritture vengono passate delle copie. Di AFS furono sviluppati vari prototipi tra per migliorarne le prestazioni e infatti nelle versioni successive furono introdotte comunicazione tramite RPC, um migliore meccanismo di gestione della cache e diminuiti i context switching dei processi Vice \cite{howard88}.
\begin{figure}[H]
  \begin{center}
    \includegraphics[scale=0.6]{afs.jpg}
    \caption{Architettura di AFS}
    \label{fg:afs.jpg}
  \end{center}
\end{figure}
Un difetto che ha portato questo file system ad essere poi lentamente rimpiazzato da NFS è stata una mancanza ottimale della gestione della coerenza dei dati in cache che non permetteva aggiornamenti in parallelo dei dati ma si era costretti a lavorare su copie dei dati e poi essere spedite nuovamente al server che, tramite meccanismi di callback, gestiva la coerenza.
\subsection{NFS}
Un approccio completamente ortogonale ad AFS fu il Network File System o conosciuto come NFS, ideato dalla Sun Microsystem nel 1984 di cui sono esistite 4 versioni, dove ognuna introdusse nuove funzionalità. La maggiore differenza con AFS risiede nell'architettura con cui è costruito; mentre nel primo c'era una netta distinzione tra client e server, in NFS la comunicazione è sempre di questo tipo ma il componente che implementa il client e il server è il medesimo e questo ha permesso di aumentarne la portabilità, permettendo così di supportare piattaforme hardware e software eterogenee. Il modulo è caricato all'interno dello spazio kernel (a differenza di AFS dove sia client che server erano implementati nello spazio utente) avendo così un aumento delle prestazioni. Il funzionamento di NFS è molto semplice: Un file system (o una porzione di esso) viene montato su NFS facendo sì che diversi client accedano alle risorse (rispettando gli opportuni privilegi) e comunicando attraverso chiamate RPC.
\begin{figure}[H]
  \begin{center}
    \includegraphics[scale=0.5]{nfs.jpg}
    \caption{Architettura di NFS}
    \label{fg:nfs.jpg}
  \end{center}
\end{figure}
\subsection{LFS}
Nel 1992 la University of California propose un log-structured file system, nominato Sprite LFS, ovvero un filesystem dove i dati e i meta dati sono scritti sequenzialmente in un buffer circolare, chiamato log.\cite{rosenblum92}
\begin{figure}[H]
  \begin{center}
    \includegraphics[scale=0.3]{lfs.png}
    \caption{Buffer di LFS}
    \label{fg:lfs.jpg}
  \end{center}
\end{figure}
Ogni sequenza di modifiche viene bufferizzata in cache e poi scritta sequenzialmente sul log (disco). Il problema principale di un log-structured file system è quello di mantenere grandi sezioni contigue di spazio libero su disco: il disco viene diviso in sezioni di dimensione fissa, chiamate segmenti. I dati vengono scritti sequenzialmente dall'inizio alla fine di ogni segmento e si raggruppano i segmenti parzialmente vuoti in un piccolo numero di segmenti pieni in modo da creare spazio contiguo per nuovi segmenti. Sebbene in grado di massimizzare la larghezza di banda usata per scrivere nuovi dati, questo file system non consente di effettuare snapshot o versioning e potrebbe creare complesse frammentazioni del file system.
\subsection{Coda}
Nel 1987 alla Carnagie Mellon University iniziò lo sviluppo di Coda, un file system distribuito. L'obiettivo era quello di realizzare un file system che sfruttasse i vantaggi offerti da un repository di dati condiviso e che fosse in grado di lavorare anche in caso di un guasto remoto o di momenti critici. L'idea principale è quella di usa una cache locale che fornisce accesso ai dati del server anche quando la connessione di rete si interrompe. Un utente legge e scrive dal file system normalmente, mentre il client "accumula" tutti i dati listati dall'utente in caso di disconnessione di rete.
Se la connessione di rete si interrompe, la cache locale del client distribuisce i dati dalla sua cache e fa il log degli aggiornamenti. Quest'operazione prende il nome di \textbf{disconnecction operation}. Alla riconnesione della rete, il client invia gli aggiornamenti loggati ai server. Una sostanziale differenza tra Coda e AFS consiste nel metodo di replica, nel caso di quest ultimo c'è un solo server di lettura/scrittura in grado di ricevere aggiornamenti mentre i rimanenti server lavorano come replice in sola lettura. Al contrario Coda consente a tutti i server di ricevere aggiornamenti, a favore di una maggiore availability.\cite{kistler92}
\begin{figure}[H]
  \begin{center}
    \includegraphics[width=\linewidth]{coda.png}
    \caption{Architettura di Coda}
    \label{fg:code.png}
  \end{center}
\end{figure}
\subsection{xFS}
Unico nel suo genere, xFS è un prototipo di un network file system senza server realizzato nel 1996 alla University of California. Lo scopo di questo lavoro era superare i problemi di un sistema con un server centralizzato, dove quest'entità costituisce un collo di bottiglia e un single point of failure. Per distribuire in maniera dinamica il processo di controllo e di archiviazione dei dati, xFS si affida a precedenti lavori di ricerca, in particolare utilizza RAID per lo storage dei dati, LFS per la composizione dei log del filesystem, Zebra, un filesystem che a sua volta combina i vantaggi di RAID e LFS (ma che nel caso di xFS sono estesi al distribuito) ed infine Dash e Alewife per la consistenza dei dati in cache. In assenza di un server centrale, in xFS i compiti svolti da quest'unità vengono delegati a quatto entità: i client, che accedono al sistema, i meta data manager che gestiscono la cache e il mapping dei metadati sul disco, gli storage server che memorizzano i dati e i cleaner che riorganizzano i dati. Un approccio del genere, seppure in grado di ottenere alte prestazioni, risulta appropriato in ambienti dove le macchine hanno fiducia l'una nell'altra \cite{anderson96}.
\begin{figure}[H]
  \begin{center}
    \includegraphics[scale=0.7]{xfs.png}
    \caption{Due possibili installazioni di xFS}
    \label{fg:xfs.jpg}
  \end{center}
\end{figure}
\subsection{GFS}
Nel 2003 BigG per fronteggiare il crescente aumento di dati gestiti realizzò un proprio file system distribuito, il Google File System. L'architettura si basa su pattern master/slave. Un cluster GFS consiste in un master e vari chunckserver. I file sono suddivisi in blocchi da 64 MB, chiamati chucks.
\begin{figure}[H]
  \begin{center}
    \includegraphics[scale=0.7]{gfs.png}
    \caption{Struttura di GFS}
    \label{fg:gfs.png}
  \end{center}
\end{figure}
Il master si occupa del controllo globale del sistema, di memorizzare i meta dati del filesystem e, inoltre, periodicamente controllare lo stato dei chuckservers tramite dei messaggi chiamati \textbf{HeartBeat}. I chuckserver provvedono alla memorizzazione dei dati, in particolare tutti i dati sono replicati su un certo numero di chuckserver, 3 di default. 
Dato che il master mantiene i metadati e gestisce la distribuzione dei file, è coinvolto nella lettura, modifica e rimozione dei chuncks. La taglia di 64 MB di un chuck può essere considerata un trade-off tra il cercare di limitare l'uso di risorse e le interazioni con il master da un lato e ottenere un alto grado di frammentazione interna dall'altro. GFS è implementato come una serie di componenti user mode eseguite su Linux. Quindi punta esclusivamente a fornire un file system distribuito lasciando il compito della gestione dei dischi ai filesystem di Linux. Basandosi su questa separazione, i chunckservers salvano ogni chunck come un file nel filesystem Linux.\cite{ghemawat03}
